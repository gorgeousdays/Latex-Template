% !Mode:: "TeX:UTF-8"
% !TEX program  = xelatex

%\documentclass{cumcmthesis}
\documentclass[withoutpreface,bwprint]{cumcmthesis} %去掉封面与编号页
\usepackage[framemethod=TikZ]{mdframed}
\usepackage{geometry}
\geometry{left=2.5cm,right=2.5cm,top=2.5cm,bottom=2.5cm}
\renewcommand{\baselinestretch}{1.0}
\usepackage{url}   % 网页链接
\usepackage{subcaption} % 子标题
\setcounter{secnumdepth}{4}
\title{数学建模}
\tihao{A}
\baominghao{4321}
\schoolname{XX大学}
\membera{ }
\memberb{ }
\memberc{ }
\supervisor{ }
\yearinput{2020}
\monthinput{08}
\dayinput{22}
\begin{document}

\maketitle
\begin{abstract}
关于摘要的一段概述、引入

\textbf{针对......的问题},我们



\textbf{针对......的问题},我们



\keywords{摘要\quad  数学建模\quad   模板\quad  公式}
\end{abstract}


\section{问题重述}
\subsection{引言}
题目中引言的概括与更改,
\subsection{问题的提出}
对于题目问题的重述


\section{问题分析}
\subsection{问题一的分析}
针对问题一,要......
\subsection{问题二的分析}
\subsection{问题三的分析}





\section{模型的假设}
\begin{itemize}
    \item 假设一
    \item 假设二
    \item ......
\end{itemize}



\section{符号说明}
\begin{tabular}{cc}
 \hline
 \makebox[0.4\textwidth][c]{符号}	&  \makebox[0.5\textwidth][c]{意义} \\ \hline
 $\theta$    & 平面A与平面B的角度 \\ \hline
 $\phi$	      & 平面A与平面C的角度 \\ \hline
\end{tabular}

\section{模型的建立}
\subsection{问题一模型的建立与求解}

\subsubsection{求解步骤1}

\subsubsection{求解步骤2}

\subsection{问题二模型的建立与求解}

\subsubsection{求解步骤1}

\subsubsection{求解步骤2}

\subsection{......}

\section{模型的灵敏度或稳定性分析}

\section{模型的推广与评价}
\subsection{模型的优点}
\subsection{模型的缺点}
\subsection{模型的推广}

%参考文献
\begin{thebibliography}{9}%宽度9
    \bibitem[1]{liuhaiyang2013latex}
    刘海洋.
    \newblock \LaTeX {}入门\allowbreak[J].
    \newblock 电子工业出版社, 北京, 2013.
    \bibitem[2]{mathematical-modeling}
    全国大学生数学建模竞赛论文格式规范 (2019 年 9 月 12 日修改).
\end{thebibliography}

\newpage
%附录
\begin{appendices}

\section{问题一所用排队算法--matlab 源程序}
\begin{lstlisting}[language=matlab]
kk=2;[mdd,ndd]=size(dd);
while ~isempty(V)
[tmpd,j]=min(W(i,V));tmpj=V(j);
for k=2:ndd
[tmp1,jj]=min(dd(1,k)+W(dd(2,k),V));
tmp2=V(jj);tt(k-1,:)=[tmp1,tmp2,jj];
end
tmp=[tmpd,tmpj,j;tt];[tmp3,tmp4]=min(tmp(:,1));
if tmp3==tmpd, ss(1:2,kk)=[i;tmp(tmp4,2)];
else,tmp5=find(ss(:,tmp4)~=0);tmp6=length(tmp5);
if dd(2,tmp4)==ss(tmp6,tmp4)
ss(1:tmp6+1,kk)=[ss(tmp5,tmp4);tmp(tmp4,2)];
else, ss(1:3,kk)=[i;dd(2,tmp4);tmp(tmp4,2)];
end;end
dd=[dd,[tmp3;tmp(tmp4,2)]];V(tmp(tmp4,3))=[];
[mdd,ndd]=size(dd);kk=kk+1;
end; S=ss; D=dd(1,:);
\end{lstlisting}

\section{问题二规划解决程序--lingo源代码}

\begin{lstlisting}[language=c]
kk=2;
[mdd,ndd]=size(dd);
while ~isempty(V)
    [tmpd,j]=min(W(i,V));tmpj=V(j);
for k=2:ndd
    [tmp1,jj]=min(dd(1,k)+W(dd(2,k),V));
    tmp2=V(jj);tt(k-1,:)=[tmp1,tmp2,jj];
end
    tmp=[tmpd,tmpj,j;tt];[tmp3,tmp4]=min(tmp(:,1));
if tmp3==tmpd, ss(1:2,kk)=[i;tmp(tmp4,2)];
else,tmp5=find(ss(:,tmp4)~=0);tmp6=length(tmp5);
if dd(2,tmp4)==ss(tmp6,tmp4)
    ss(1:tmp6+1,kk)=[ss(tmp5,tmp4);tmp(tmp4,2)];
else, ss(1:3,kk)=[i;dd(2,tmp4);tmp(tmp4,2)];
end;
end
    dd=[dd,[tmp3;tmp(tmp4,2)]];V(tmp(tmp4,3))=[];
    [mdd,ndd]=size(dd);
    kk=kk+1;
end;
S=ss;
D=dd(1,:);
 \end{lstlisting}
\section{问题三所用Python源代码}
\begin{lstlisting}[language=Python]
print("数学建模论文模板")
\end{lstlisting}
\end{appendices}

\end{document} 